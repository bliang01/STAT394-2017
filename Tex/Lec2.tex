\documentclass[usenames,dvipsnames,12pt,compress, final]{beamer}\usepackage[]{graphicx}\usepackage[]{color}
\usepackage{alltt} 
\let\Tiny=\tiny
\usepackage{eqnarray,amsmath}
% \usepackage{wasysym}
\usepackage{mathtools}
\usepackage{multirow}
\usepackage{booktabs}
\usepackage{tabularx}
\usepackage{array}
\usepackage{multirow}
% \usepackage{bigstrut}
\usepackage{graphicx}
\usepackage[round]{natbib}
\usepackage{bm}
\usepackage{tikzsymbols}
\setbeamersize{text margin left=15pt,text margin right=10pt}
\usepackage{tikzsymbols}
\usepackage{textcomp}
\usepackage{parskip}
\setbeamertemplate{navigation symbols}{}    
\setbeamertemplate{footline}[frame number]{}
\usepackage{soul}
\usetheme{Singapore}
\usecolortheme{lily}
%Misc Commands
\newcommand{\mbf}{\mathbf}
\newcommand{\lexp}{$\overset{\mbox{\tiny 0}}{e}$}
\newenvironment{wideitemize}{\itemize\addtolength{\itemsep}{5pt}}{\enditemize}


\newcommand{\bx}{{\bm x}}
\newcommand{\bX}{{\bm X}}
\newcommand{\by}{{\bm y}}
\newcommand{\bY}{{\bm Y}}
\newcommand{\bW}{{\bm W}}
\newcommand{\bG}{{\bm G}}
\newcommand{\bR}{{\bm R}}
\newcommand{\bZ}{{\bm Z}}
\newcommand{\bV}{{\bm V}}
\newcommand{\bL}{{\bm L}}
\newcommand{\bz}{{\bm z}}
\newcommand{\be}{{\bm e}}
\newcommand{\bgamma}{{\bm \gamma}}
\newcommand{\bbeta}{{\bm \beta}}
\newcommand{\balpha}{{\bm \alpha}}
\newcommand{\bSigma}{{\bm \Sigma}}
\newcommand{\bmu}{{\bm \mu}}
\newcommand{\btheta}{{\bm \theta}}
\newcommand{\bepsilon}{{\bm \epsilon}}
\newcommand{\bone}{{\bm 1}}
\newcommand{\bzero}{{\bm 0}}
\newcommand{\bC}{{\bm C}}
\newcommand{\bI}{{\bm I}}
\newcommand{\bA}{{\bm A}}
\newcommand{\bB}{{\bm B}}
\newcommand{\bQ}{{\bm Q}}
\newcommand{\bS}{{\bm S}}
\newcommand{\bD}{{\bm D}}
\newcommand{\cQ}{\mathcal{Q}}
\newcommand{\cR}{\mathcal{R}}
\newcommand{\cU}{\mathcal{U}}
\newcommand{\cI}{\mathcal{I}}
\newcommand{\cL}{\mathcal{L}}
\newcommand{\RR}{\mathbb{R}}
\newcommand{\orange}{\textcolor{Orange}}
\newcommand{\green}{\textcolor{green}}
\newcommand{\blue}{\textcolor{blue}}
\newcommand{\red}{\textcolor{red}}
\newcommand{\purple}{\textcolor{purple}}
\newcommand{\gray}{\textcolor{gray}}
\newcommand{\ok}{\nonumber}

% Adjust vertical spacing in lists
\makeatletter
\def\@listi{\leftmargin\leftmargini
            \topsep 		8\p@ \@plus2\p@ \@minus2.5\p@
            \parsep 		0\p@
            \itemsep	5\p@ \@plus2\p@ \@minus3\p@}
\let\@listI\@listi
\def\@listii{\leftmargin\leftmarginii
              \topsep    6\p@ \@plus1\p@ \@minus2\p@
              \parsep    0\p@ \@plus\p@
              \itemsep  3\p@ \@plus2\p@ \@minus3\p@}
\def\@listiii{\leftmargin\leftmarginiii
              \topsep    3\p@ \@plus1\p@ \@minus2\p@
              \parsep    0\p@ \@plus\p@
              \itemsep  2\p@ \@plus2\p@ \@minus3\p@}
\makeatother
% Dealing with fraile envrionment of beamer with codes
\newenvironment{xframe}[2][]
  {\begin{frame}[fragile,environment=xframe,#1]
  \frametitle{#2}}
  {\end{frame}}



\title{Stat 394 Probability I}
\subtitle{Lecture 2}

\author[]{Richard Li}
\date{\today}
\IfFileExists{upquote.sty}{\usepackage{upquote}}{}
\begin{document}
% \renewcommand{\itemize}[1][<+(1)->]{\olditemize[#1]}

\maketitle

%================================================================%
\section{Combinatorics}
\stepcounter{subsection}
\frame{
	\frametitle{Review}
}
\frame{
\frametitle{Example}
	You are at a Poke place, from $5$ kinds of fish and $10$ choices of toppings, how many different combinations consisting of $3$ fish and $5$ toppings can be formed (assume you can't choose the same fish more than once). What if $2$ of the toppings are very spicy, and you don't want to have them both together?
	\vspace{4cm}

}
\frame{
	\frametitle{Multinomial coefficients}
	A set of $n$ distinct items is to be divided into $r$ distinct groups of size $n_1, n_2, ..., n_r$, where $\sum_{i=1}^r n_i = n$. How many different divisions are possible?
	\vspace{3cm}
}

\frame{
	\frametitle{Example}
	In a knockout tournament involving $64$ players. In each round, the players are divided into pairs, and the winners go on the next round.  How many possible outcomes are there for the first round? 
	\vspace{4cm}
}
\frame{
	\frametitle{Example}
	
}


\frame{
	\frametitle{Example}
	
}
\frame{
	\frametitle{Summary}
	\begin{itemize}
		
		\item Permutation and combination are strongly related, as we have seen in examples.
		\item We will see more of these in the chapters to come.
		\item Read Chapter 1 on your own (especially multinomial theorem in Sec 1.5).
		\item Homework 1 due next Monday.  
	\end{itemize}
}
%================================================================%
\section{Sample space and events}
\stepcounter{subsection}
\frame{
	\frametitle{Introduction}
	\begin{itemize}
		\item Before we start talking about probability, we need to define it.
		\item To do that, we need to review some set theory and definitions.
	\end{itemize}
}

\frame{
	\frametitle{Sample space}
	\begin{itemize}
		\item Experiment
				\vspace{1.5cm}
	
		\item Sample space
				\vspace{1.5cm}
		\item Event
				\vspace{1.5cm}
	\end{itemize}
}
\frame{
	\frametitle{Examples}
}
 
\frame{
	\frametitle{Set theory}
	Suppose that $E$ and $F$ are subsets of a set $S$
	\begin{itemize}
		\item The union of events $E$ and $F$ is
		\vspace{1cm}
		\item  The intersection of events $E$ and $F$ is
		\vspace{1cm}
		\item  $E$ and $F$ are disjoint (or mutually exclusive) if
		\vspace{1cm}
	\end{itemize}
}
\frame{
	\frametitle{Set theory}
	\begin{itemize}
		\item The complement of $E$ is
		\vspace{1cm}
		\item  The set difference of $E$ and $F$ is
		\vspace{1cm}
		\item  Venn diagram
		\vspace{3cm}
	\end{itemize}
}
\frame{
	\frametitle{Example}
	Let $S = \RR^2$, and two events
	\[
		E = \{(x, y) \in S : 0 < x < 3, 0 < y < 3\}
	\]
	\[
		F = \{(x, y) \in S : 2 < x < 4, 0 < y < 4\}
	\]
	\vspace{3cm}
}
\frame{
	\frametitle{Basic rules}
	\begin{itemize}
		\item Commutative laws
		\vspace{2cm}
		\item Associative laws
		\vspace{2cm}
		\item Distributive laws
		\vspace{2cm}
	\end{itemize}
}
\frame{
	\frametitle{Distributive laws}
	
}
\frame{
	\frametitle{DeMorgan's laws}
	
}
\frame{
	\frametitle{DeMorgan's laws}
	
}

%================================================================%
\section{Probability}
\stepcounter{subsection}
\frame{
	\frametitle{How do we define probability?}
	\pause
	Example: Roll a die 4 times, what is the probability that you get different numbers?
	\vspace{4cm}
}

\frame{
	\frametitle{Why is the example too simple?}
}

\frame{
	\frametitle{Probability space}
	A probability space is a triple $(S, \mathcal{F}, P)$ (FYI, in many other books, notation $S$ is replaced by $\Omega$)
	\vspace{4.5cm}
}


\frame{
	\frametitle{Probability axioms}
	
}

\frame{
	\frametitle{Example}
	\begin{itemize}
		\item Let $S = \{1, 2, ... \}$
		\item For any $E \subset S$, 
		\[
			P(E) = \sum_{i \in E} \frac{1}{2^i}
		\]
		\item Is it a probability space?
	\end{itemize}
	\vspace{3cm}
}

\frame{
	\frametitle{Propositions}
}
\frame{
	\frametitle{Propositions}
}
\frame{
	\frametitle{Propositions}
}


\frame{
	\frametitle{Example}
	Pick an integer in $[1,1000]$ at random. Compute the probability that it is divisible neither by 12 nor by 15.
	\vspace{5cm}
}
\frame{
	\frametitle{Example}
	Sit 3 men and 4 women at random in a row. What is the probability that either all the men or all the women end up sitting together?
	\vspace{5cm}
}

\frame{
	\frametitle{Example}
	A group of 3 Democrats, 4 Republicans, and 5 Independents is seated at random around a table. Compute the probability that at least one of the three groups ends up sitting together.
	\vspace{5cm}
}
\frame{
	\frametitle{Example}
	A large company with n employees has a scheme according to which each employee buys a Christmas gift and the gifts are then distributed at random to the employees. What is the probability that someone gets his or her own gift?
	\vspace{5cm}
}
\frame{
	\frametitle{Example}
	
}
%================================================================%
\section{Series}
\stepcounter{subsection}
\frame{
	\frametitle{Just a review}
	A good time to recall some common mathematical series:
	\begin{eqnarray}\ok
		\exp(x)&=& \sum_{n = 0}^\infty \frac{x^n}{n!}
	\\\ok
		log(1 - x)&=& -\sum_{n = 1}^\infty \frac{x^n}{n}
	\\\ok
		log(1 + x) &=& \sum_{n = 1}^\infty (-1)^{n+1} \frac{x^n}{n}
	\\\ok
		\frac{1}{1 - x} &=& \sum_{n = 0}^\infty x^n
	\end{eqnarray}
}
\frame{
	\frametitle{Binomial series, or binomial theorem}
	\[
		(x + y)^n = \sum_{k = 0}^n {n \choose k} x^k y^{n - k}
	\]
	\vspace{5cm}


}
\frame{
	\frametitle{Binomial series, or binomial theorem}


}

\frame{
	\frametitle{Example}
	How many subsets are there of a set consisting of $n$ elements?
	\vspace{5cm}
}


\end{document}