\documentclass[usenames,dvipsnames,12pt,compress, final, handout]{beamer}\usepackage[]{graphicx}\usepackage[]{color}
\usepackage{alltt} 
\let\Tiny=\tiny
\usepackage{eqnarray,amsmath}
% \usepackage{wasysym}
\usepackage{mathtools}
\usepackage{multirow}
\usepackage{booktabs}
\usepackage{tabularx}
\usepackage{array}
\usepackage{multirow}
% \usepackage{bigstrut}
\usepackage{graphicx}
\usepackage[round]{natbib}
\usepackage{bm}
\usepackage{tikzsymbols}
\setbeamersize{text margin left=15pt,text margin right=10pt}
\usepackage{tikzsymbols}
\usepackage{textcomp}
\usepackage{parskip}
\setbeamertemplate{navigation symbols}{}    
\setbeamertemplate{footline}[frame number]{}
\usepackage{soul}
\usetheme{Singapore}
\usecolortheme{lily}
%Misc Commands
\newcommand{\mbf}{\mathbf}
\newcommand{\lexp}{$\overset{\mbox{\tiny 0}}{e}$}
\newenvironment{wideitemize}{\itemize\addtolength{\itemsep}{5pt}}{\enditemize}


\newcommand{\bx}{{\bm x}}
\newcommand{\bX}{{\bm X}}
\newcommand{\by}{{\bm y}}
\newcommand{\bY}{{\bm Y}}
\newcommand{\bW}{{\bm W}}
\newcommand{\bG}{{\bm G}}
\newcommand{\bR}{{\bm R}}
\newcommand{\bZ}{{\bm Z}}
\newcommand{\bV}{{\bm V}}
\newcommand{\bL}{{\bm L}}
\newcommand{\bz}{{\bm z}}
\newcommand{\be}{{\bm e}}
\newcommand{\bgamma}{{\bm \gamma}}
\newcommand{\bbeta}{{\bm \beta}}
\newcommand{\balpha}{{\bm \alpha}}
\newcommand{\bSigma}{{\bm \Sigma}}
\newcommand{\bmu}{{\bm \mu}}
\newcommand{\btheta}{{\bm \theta}}
\newcommand{\bepsilon}{{\bm \epsilon}}
\newcommand{\bone}{{\bm 1}}
\newcommand{\bzero}{{\bm 0}}
\newcommand{\bC}{{\bm C}}
\newcommand{\bI}{{\bm I}}
\newcommand{\bA}{{\bm A}}
\newcommand{\bB}{{\bm B}}
\newcommand{\bQ}{{\bm Q}}
\newcommand{\bS}{{\bm S}}
\newcommand{\bD}{{\bm D}}
\newcommand{\cQ}{\mathcal{Q}}
\newcommand{\cR}{\mathcal{R}}
\newcommand{\cU}{\mathcal{U}}
\newcommand{\cI}{\mathcal{I}}
\newcommand{\cL}{\mathcal{L}}
\newcommand{\RR}{\mathbb{R}}
\newcommand{\orange}{\textcolor{Orange}}
\newcommand{\green}{\textcolor{green}}
\newcommand{\blue}{\textcolor{blue}}
\newcommand{\red}{\textcolor{red}}
\newcommand{\purple}{\textcolor{purple}}
\newcommand{\gray}{\textcolor{gray}}
\newcommand{\ok}{\nonumber}

% Adjust vertical spacing in lists
\makeatletter
\def\@listi{\leftmargin\leftmargini
            \topsep     8\p@ \@plus2\p@ \@minus2.5\p@
            \parsep     0\p@
            \itemsep  5\p@ \@plus2\p@ \@minus3\p@}
\let\@listI\@listi
\def\@listii{\leftmargin\leftmarginii
              \topsep    6\p@ \@plus1\p@ \@minus2\p@
              \parsep    0\p@ \@plus\p@
              \itemsep  3\p@ \@plus2\p@ \@minus3\p@}
\def\@listiii{\leftmargin\leftmarginiii
              \topsep    3\p@ \@plus1\p@ \@minus2\p@
              \parsep    0\p@ \@plus\p@
              \itemsep  2\p@ \@plus2\p@ \@minus3\p@}
\makeatother
% Dealing with fraile envrionment of beamer with codes
\newenvironment{xframe}[2][]
  {\begin{frame}[fragile,environment=xframe,#1]
  \frametitle{#2}}
  {\end{frame}}



\title{Stat 394 Probability I}
\subtitle{Lecture 7}

\author[]{Richard Li}
\date{\today}
\IfFileExists{upquote.sty}{\usepackage{upquote}}{}
\begin{document}
% \renewcommand{\itemize}[1][<+(1)->]{\olditemize[#1]}

\maketitle

%================================================================%
\section{Sum of random variable}
\stepcounter{subsection}
\frame{
  \frametitle{More about expectation and variance operation}
}
\frame{
  \frametitle{Expectation of sums}
  For random variables $X_1, X_2, ..., X_n$, $E(\sum_{i=1}^n X_i) = \sum_{i=1}^n E(X_i)$
  \vspace{6cm}
}
 
%================================================================%
\section{Geometric distribution}
\stepcounter{subsection}
\frame{
  \frametitle{Geometric distribution}
  Suppose independent trials, each of which results in a \textit{success} with probability $p$ and \textit{failure} with probability $1-p$, are performed until a success occurs. Let $X$ represent the number of trials required, then $X$ is said to be a \textbf{geometric random variable} with parameter $p$, i.e., $X \sim Geom(p)$, and
  \[
      p(X = k) = (1 - p)^{k-1}p, k = 1, 2, ...
  \]
}
 
\frame{
  \frametitle{Expectation and variance}
}

\frame{
  \frametitle{Coupon collector problem}
  Suppose you are at the entrance of Odegaard library and ask the birthday of each person entering the library. How many people do you expect that you need to ask, in order to collect all $N = 365$ different days (ignoring Feb 29).

  \vspace{4cm}
}
%================================================================%
\section{Negative binomial distribution}
\stepcounter{subsection} 
\frame{
  \frametitle{Negative binomial distribution}
  Suppose independent trials, each of which results in a \textit{success} with probability $p$ and \textit{failure} with probability $1-p$, are performed until $r$ success occurs. Let $X$ represent the number of trials required, then $X$ is said to be a \textbf{negative binomial random variable} with parameter $(r, p)$, i.e., $X \sim NB(r, p)$, and
  \[
      p(X = k) = {k - 1 \choose r - 1}p^r(1 - p)^{k-r}, k = r, r + 1, ...
  \]
}
\frame{
  \frametitle{Expectation and variance}
}
 
\frame{
\frametitle{The Banach match problem}
A mathematician carries two matchboxes, each originally containing $n$ matches. Each time he needs a match, he is equally likely to take it from either box. What is the probability that, upon reaching for a box and finding it empty, there are exactly $k$ matches still in the other box? Here, $0 \leq k \leq n$.
\vspace{3cm}
}
%================================================================%
\section{Hypergeometric distribution}
\stepcounter{subsection}
\frame{
  \frametitle{Hypergeometric distribution}
  Suppose a sample of size $n$ is to be chosen randomly (without replacement) from an urn containing $N$ balls, of which $m$ are white and $N-m$ are black. Let $X$ denote the number of white balls selected, then $X$ is said to be a \textbf{hypergeometric random variable},
  \[
      p(X = k) = \frac{{m \choose k}{N - m \choose n - k}}{{N \choose n}}, k = 0, 1, ..., n
  \]
  and we can show (see textbook) that
  \[
    E(X) = \frac{mn}{N}, \;\; Var(X) = \frac{mn}{N}(1-\frac{m}{N})(1-\frac{n-1}{N-1})
  \]
}

\end{document}