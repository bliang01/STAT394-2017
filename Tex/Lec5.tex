\documentclass[usenames,dvipsnames,12pt,compress, final, handout]{beamer}\usepackage[]{graphicx}\usepackage[]{color}
\usepackage{alltt} 
\let\Tiny=\tiny
\usepackage{eqnarray,amsmath}
% \usepackage{wasysym}
\usepackage{mathtools}
\usepackage{multirow}
\usepackage{booktabs}
\usepackage{tabularx}
\usepackage{array}
\usepackage{multirow}
% \usepackage{bigstrut}
\usepackage{graphicx}
\usepackage[round]{natbib}
\usepackage{bm}
\usepackage{tikzsymbols}
\setbeamersize{text margin left=15pt,text margin right=10pt}
\usepackage{tikzsymbols}
\usepackage{textcomp}
\usepackage{parskip}
\setbeamertemplate{navigation symbols}{}    
\setbeamertemplate{footline}[frame number]{}
\usepackage{soul}
\usetheme{Singapore}
\usecolortheme{lily}
%Misc Commands
\newcommand{\mbf}{\mathbf}
\newcommand{\lexp}{$\overset{\mbox{\tiny 0}}{e}$}
\newenvironment{wideitemize}{\itemize\addtolength{\itemsep}{5pt}}{\enditemize}


\newcommand{\bx}{{\bm x}}
\newcommand{\bX}{{\bm X}}
\newcommand{\by}{{\bm y}}
\newcommand{\bY}{{\bm Y}}
\newcommand{\bW}{{\bm W}}
\newcommand{\bG}{{\bm G}}
\newcommand{\bR}{{\bm R}}
\newcommand{\bZ}{{\bm Z}}
\newcommand{\bV}{{\bm V}}
\newcommand{\bL}{{\bm L}}
\newcommand{\bz}{{\bm z}}
\newcommand{\be}{{\bm e}}
\newcommand{\bgamma}{{\bm \gamma}}
\newcommand{\bbeta}{{\bm \beta}}
\newcommand{\balpha}{{\bm \alpha}}
\newcommand{\bSigma}{{\bm \Sigma}}
\newcommand{\bmu}{{\bm \mu}}
\newcommand{\btheta}{{\bm \theta}}
\newcommand{\bepsilon}{{\bm \epsilon}}
\newcommand{\bone}{{\bm 1}}
\newcommand{\bzero}{{\bm 0}}
\newcommand{\bC}{{\bm C}}
\newcommand{\bI}{{\bm I}}
\newcommand{\bA}{{\bm A}}
\newcommand{\bB}{{\bm B}}
\newcommand{\bQ}{{\bm Q}}
\newcommand{\bS}{{\bm S}}
\newcommand{\bD}{{\bm D}}
\newcommand{\cQ}{\mathcal{Q}}
\newcommand{\cR}{\mathcal{R}}
\newcommand{\cU}{\mathcal{U}}
\newcommand{\cI}{\mathcal{I}}
\newcommand{\cL}{\mathcal{L}}
\newcommand{\RR}{\mathbb{R}}
\newcommand{\orange}{\textcolor{Orange}}
\newcommand{\green}{\textcolor{green}}
\newcommand{\blue}{\textcolor{blue}}
\newcommand{\red}{\textcolor{red}}
\newcommand{\purple}{\textcolor{purple}}
\newcommand{\gray}{\textcolor{gray}}
\newcommand{\ok}{\nonumber}

% Adjust vertical spacing in lists
\makeatletter
\def\@listi{\leftmargin\leftmargini
            \topsep     8\p@ \@plus2\p@ \@minus2.5\p@
            \parsep     0\p@
            \itemsep  5\p@ \@plus2\p@ \@minus3\p@}
\let\@listI\@listi
\def\@listii{\leftmargin\leftmarginii
              \topsep    6\p@ \@plus1\p@ \@minus2\p@
              \parsep    0\p@ \@plus\p@
              \itemsep  3\p@ \@plus2\p@ \@minus3\p@}
\def\@listiii{\leftmargin\leftmarginiii
              \topsep    3\p@ \@plus1\p@ \@minus2\p@
              \parsep    0\p@ \@plus\p@
              \itemsep  2\p@ \@plus2\p@ \@minus3\p@}
\makeatother
% Dealing with fraile envrionment of beamer with codes
\newenvironment{xframe}[2][]
  {\begin{frame}[fragile,environment=xframe,#1]
  \frametitle{#2}}
  {\end{frame}}



\title{Stat 394 Probability I}
\subtitle{Lecture 5}

\author[]{Richard Li}
\date{\today}
\IfFileExists{upquote.sty}{\usepackage{upquote}}{}
\begin{document}
% \renewcommand{\itemize}[1][<+(1)->]{\olditemize[#1]}

\maketitle

%================================================================%
\section{Random variables}
\stepcounter{subsection}
\frame{
  \frametitle{Example}
  \begin{itemize}
    \item Toss a coin 10 times and let $X$ be the number of Heads.
    \item Choose a random point in the unit square $\{(x,y) : 0 \leq x, y \leq 1\}$ and let $X$ be its distance from the origin.
    \item Choose a random person in a class and let $X$ be the height of the person, in inches.
  \end{itemize}
}
\frame{
  \frametitle{Random variables}
  A \textbf{random variable} is a number whose value depends upon the outcome of a random experiment.

  \vspace{1cm}

  A random variable $X$ is a real-valued function on sample space $S$:
  \[
    X : S \rightarrow \mathcal{R}
  \]
   
}

\frame{
\frametitle{Example}
  
}

\frame{
  \frametitle{Discrete random variables}

  A \textbf{discrete random variable} $X$ has finitely or countably many values $x_i, i = 1, 2, . . .$

  \vspace{1cm}
  
  $p(x_i) = P(X = x_i)$ with $i = 1,2,...$ is called the \textbf{probability mass function} (or p.m.f) of $X$. 

  \vspace{3cm}
}

\frame{
  \frametitle{Using r.v. to solve probability problems}
  4 balls are to be randomly selected without replacement from an urn containing 10 balls numbered 1 through 10. If we bet that at least one of the balls that are drawn has a number as large as or larger than 7, what is the probability that we win the bet?

  \vspace{3cm}
}

\frame{
  \frametitle{Bernoulli random variables}
  Suppose an experiment where the outcome can be considered as either a \textit{success} or a \textit{failure}, let $X=1$ when outcome is success and $X=0$ when outcome is failure, then $X$ is said to be a \textbf{Bernoulli random variable} with parameter $p$. 

}

\frame{
  \frametitle{Binomial random variables}
  Suppose for $n$ independent trials, each of which results in a \textit{success} with probability $p$ and \textit{failure} with probability $1-p$, and let $X$ represent the number of success that occur in the $n$ trials, then $X$ is said to be a \textbf{binomial random variable} with parameter $(n, p)$.
}

\frame{
  \frametitle{Bernoulli and Binomial distribution}
}

\frame{
  \frametitle{Cumulative distribution function (CDF)}
}
\frame{
  \frametitle{Example}
  }
\frame{
  \frametitle{Example}
  A communication system consists of $n$ components, each of which will, independently, function with probability $p$. The total system will be able to operate effectively if at least one-half of its components function. In general, when is a $(2k + 1)$-component system better than a $(2k-1)$-component system?

  \vspace{3cm}
}
%================================================================%
\section{Expectation and variance}
\stepcounter{subsection}

\frame{
  \frametitle{First, why do we want to study random variables}
  A quote:

   ``Once you get what a random variable is, it can be hard to explain. Now that I understand what a random variable is, it is difficult to remember what was difficult to understand about it''
}
\frame{
  \frametitle{First, why do we want to study random variables}
 But seriously, the way I see the advantage of using random variables
 \begin{itemize}
     \item Describe more complicated events without cumbersome notations
     \item Focus on the quantities we care about
     \item Consider problems beyond calculating probabilities
     \item Handle continuity and infinity more easily
   \end{itemize}  
}

\frame{
  \frametitle{Expectation}
  Assume $X$ is a discrete random variable with possible values $x_i, i = 1, 2, ...$. Then the \textbf{expected value}, or \textbf{expectation}, of $X$ is
  \[
      E(X) \mbox{ or } EX = \sum_i x_i p(x_i)
  \]
}
\frame{
  \frametitle{Example}
  A box contains $5$ red balls and $3$ blue balls. Two balls are withdrawn randomly, and if they are of the same color, you win \$2, otherwise you lose \$1. What is the amount you expect to win?
  \vspace{4cm}
}
\frame{
  \frametitle{Chocolate store}
  A candy store has $M$ boxes of chocolates. The number of chocolates in each box vary. Suppose for $k = 1, 2, ..., N$, there are $n_k$ boxes containing $k$ chocolates, and $\sum_{k=1}^N n_k = M$. The store owner keeps a magic book where all the chocolates labeled $1, 2, ..., \sum_{k=1}^N kn_k$ are listed. 

  The store owner lets you to take one box of chocolate for free, and gives you two options:
  \begin{enumerate}
    \item Take one box without knowing how many chocolates are in the box.
    \item Pick one chocolate from the book, and you can take the box containing that chocolate.
  \end{enumerate}
}
\frame{
  \frametitle{Chocolate store}
}

% \frame{
% \frametitle{Two envelop problem}
% }
\frame{
  \frametitle{Expectation of function of a r.v.}
  Assume $X$ is a discrete random variable with possible values $x_i, i = 1, 2, ...$. Then the \textbf{expected value}, or \textbf{expectation}, of $g(X)$, where $g$ is some real-valued function, is
  \[
      E(g(X)) \mbox{ or } Eg(X) = \sum_i g(x_i) p(x_i)
  \]
}

\frame{
  \frametitle{Example}
}

\frame{
  \frametitle{Variance}
  If $X$ is a random variable, we say the \textbf{variance} of $X$ is defined by
  \[
    Var(X) = E((X - E(X))^2)
  \]
  and the square root of the variance is usually called the \textbf{standard deviation} of $X$ 
}
\frame{
  \frametitle{Example}
    A box contains $5$ red balls and $3$ blue balls. There are two games offered to you. Both games ask you to take one ball out,
    \begin{itemize}
       \item Game 1: if it is a red ball, you win \$2, otherwise you lose \$2.
       \item Game 2: if it is a red ball, you win \$5, otherwise you lose \$7.
     \end{itemize}  
  \vspace{4cm}
}
\frame{
  \frametitle{Alternative formula for variance}
}
\frame{
  \frametitle{Expectation and variance under linear transformation}
}
\frame{
  \frametitle{Binomial random variable revisit}
}
\frame{
  \frametitle{Binomial random variable revisit}
}
\frame{
  \frametitle{Example}
  Let $X$ be the number of Heads in 50 tosses of a fair coin. Determine $E(X)$, $Var(X)$ and $P(X \leq 10)$.
  \vspace{5cm}
}
\frame{
  \frametitle{Newsboy}
  A newsboy buy papers at 10 cents and sells them at 15 cents. However, he is not allowed to return unsold papers. If his daily demand is a binomial random variable with $n=10$ and $p=\frac{1}{3}$, approximately how many papers should he purchase so as to maximize his expected profit?
  \vspace{4cm}
}
\frame{
  \frametitle{Newsboy}
 
}

\frame{
  \frametitle{Binomial p.m.f}
  \url{http://shiny.albany.edu/stat/binomial/}
  \vspace{8cm}
}

\frame{
  \frametitle{Birthday problem revisit}
}
\frame{
  \frametitle{Birthday problem revisit}
}
\frame{
  \frametitle{Expectation of sums}
}

\end{document}