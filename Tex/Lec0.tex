\documentclass[usenames,dvipsnames,11pt,compress, final, handout]{beamer}\usepackage[]{graphicx}\usepackage[]{color}
\usepackage{alltt} 
\let\Tiny=\tiny
\usepackage{eqnarray,amsmath}
% \usepackage{wasysym}
\usepackage{mathtools}
\usepackage{multirow}
\usepackage{booktabs}
\usepackage{tabularx}
\usepackage{array}
\usepackage{multirow}
% \usepackage{bigstrut}
\usepackage{graphicx}
\usepackage[round]{natbib}
\usepackage{bm}
\usepackage{tikzsymbols}
\setbeamersize{text margin left=15pt,text margin right=10pt}
\usepackage{tikzsymbols}
\usepackage{textcomp}
\usepackage{parskip}
\setbeamertemplate{navigation symbols}{}    
\setbeamertemplate{footline}[frame number]{}
\usepackage{soul}
\usetheme{Singapore}
\usecolortheme{lily}
%Misc Commands
\newcommand{\mbf}{\mathbf}
\newcommand{\lexp}{$\overset{\mbox{\tiny 0}}{e}$}
\newenvironment{wideitemize}{\itemize\addtolength{\itemsep}{5pt}}{\enditemize}


\newcommand{\bx}{{\bm x}}
\newcommand{\bX}{{\bm X}}
\newcommand{\by}{{\bm y}}
\newcommand{\bY}{{\bm Y}}
\newcommand{\bW}{{\bm W}}
\newcommand{\bG}{{\bm G}}
\newcommand{\bR}{{\bm R}}
\newcommand{\bZ}{{\bm Z}}
\newcommand{\bV}{{\bm V}}
\newcommand{\bL}{{\bm L}}
\newcommand{\bz}{{\bm z}}
\newcommand{\be}{{\bm e}}
\newcommand{\bgamma}{{\bm \gamma}}
\newcommand{\bbeta}{{\bm \beta}}
\newcommand{\balpha}{{\bm \alpha}}
\newcommand{\bSigma}{{\bm \Sigma}}
\newcommand{\bmu}{{\bm \mu}}
\newcommand{\btheta}{{\bm \theta}}
\newcommand{\bepsilon}{{\bm \epsilon}}
\newcommand{\bone}{{\bm 1}}
\newcommand{\bzero}{{\bm 0}}
\newcommand{\bC}{{\bm C}}
\newcommand{\bI}{{\bm I}}
\newcommand{\bA}{{\bm A}}
\newcommand{\bB}{{\bm B}}
\newcommand{\bQ}{{\bm Q}}
\newcommand{\bS}{{\bm S}}
\newcommand{\bD}{{\bm D}}
\newcommand{\cQ}{\mathcal{Q}}
\newcommand{\cU}{\mathcal{U}}
\newcommand{\cI}{\mathcal{I}}
\newcommand{\cL}{\mathcal{L}}
\newcommand{\orange}{\textcolor{Orange}}
\newcommand{\green}{\textcolor{green}}
\newcommand{\blue}{\textcolor{blue}}
\newcommand{\red}{\textcolor{red}}
\newcommand{\purple}{\textcolor{purple}}
\newcommand{\gray}{\textcolor{gray}}
\newcommand{\ok}{\nonumber}

% Adjust vertical spacing in lists
\makeatletter
\def\@listi{\leftmargin\leftmargini
            \topsep 		8\p@ \@plus2\p@ \@minus2.5\p@
            \parsep 		0\p@
            \itemsep	5\p@ \@plus2\p@ \@minus3\p@}
\let\@listI\@listi
\def\@listii{\leftmargin\leftmarginii
              \topsep    6\p@ \@plus1\p@ \@minus2\p@
              \parsep    0\p@ \@plus\p@
              \itemsep  3\p@ \@plus2\p@ \@minus3\p@}
\def\@listiii{\leftmargin\leftmarginiii
              \topsep    3\p@ \@plus1\p@ \@minus2\p@
              \parsep    0\p@ \@plus\p@
              \itemsep  2\p@ \@plus2\p@ \@minus3\p@}
\makeatother
% Dealing with fraile envrionment of beamer with codes
\newenvironment{xframe}[2][]
  {\begin{frame}[fragile,environment=xframe,#1]
  \frametitle{#2}}
  {\end{frame}}



\title{Stat 394 Probability I}
\subtitle{Lecture 0}

\author[]{Richard Li}
\date{\today}
\IfFileExists{upquote.sty}{\usepackage{upquote}}{}
\begin{document}
% \renewcommand{\itemize}[1][<+(1)->]{\olditemize[#1]}

\maketitle
%================================================================%
\section{Logistics}
\stepcounter{subsection}
\frame{
	\frametitle{About this course}
	\begin{itemize}
		\item \blue{Instructor:} Z. Richard Li\\
			\begin{itemize}
			  	\item {\bf Office:} C-14G Padelford
			  	\item {\bf Office Hours:} M 12:00-2:00 PM, in C14-G Padelford (may move to C-14A next door when office gets crowded)
			  	\item {\bf E-mail:} lizehang@uw.edu\\
			  \end{itemize}  
		\pause	  
		\item \blue{TA:} Anna Green 
			\begin{itemize}
				\item {\bf TA E-mail:} greena64@uw.edu 
				\item {\bf TA Office Hours:} TBD 
			\end{itemize}
		\pause
		\item \blue{Course website:} Canvas
			\begin{itemize}
					\item Syllabus, learning objectives, schedule, lecture slides, etc.
					\item Important announcements!
					\item Homeworks. \red{\textit{(first HW due next Monday!)}}
					\item Discussion thread for homework.
					\item Please do not e-mail homework questions to me or to the TA. Use the discussion board instead.
				\end{itemize}	
	\end{itemize}
}
\frame{
	\frametitle{Course expectations}
	\begin{itemize}
		\item \blue{Pre-reqs:}
				\begin{itemize}
					\item Calculus at the level of MATH 124, 125, and 126.
					\item Basic linear algebra, familiar with matrix notations.
					\item Multivariate calculus if you are also taking 395 in term B (not required for 394).
				\end{itemize}
		\item \blue{Textbook:} \textit{A First Course in Probability}, 9th ed. by Sheldon Ross
			\begin{itemize}
				\item Previous editions are fine for learning the material.
				\item \red{Homework problems will be from the 9th edition}.
				\item Self-test problems after each chapters are good exercise.
				\item We will cover mostly Chapter 2 - 4 and part of chapter 5 in this term.
				\item You need to read and study Chapter 1 yourself (We will briefly introduce some of it today).	
			\end{itemize}
	\end{itemize}
}


\frame{
\frametitle{Evaluation}
\begin{itemize}
\item \blue{Participation (10\%)}
\item \blue{Homework (40\%):} There will be 5 homework sets. NO LATE HOMEWORKS ACCEPTED.
\item \blue{Final (50\%):} closed book, closed notes; in class final exam on July 19, JHN 026. 
\end{itemize} 
}

\frame{
	\frametitle{Participation}
	\begin{itemize}
		\item \blue{Course structures}
				\begin{itemize}
					\item Lecture: Monday, Wednesday, and first hour on Friday.
					\item Problem session: second hour on Friday.\pause
						\begin{itemize}
							\item You will work on selected problems in groups
							\item And share your solutions (verbally or on the white board)
						\end{itemize}
				\end{itemize}
		\pause		
		\item \blue{Participation credits}
			\begin{itemize}
			\item Participation in the Friday problem session. 
			\item Ask questions.
			\item Answer and explain questions from your peers.
			\item Participate in \textit{``probability in the news''}. At least one news post is required for getting the participation credits.
		\end{itemize} 
	\end{itemize}
}

\frame{
	\frametitle{Probability in the news}
	\includegraphics[width = .9\textwidth]{fun/538.png}

	{\small
	source: \url{https://fivethirtyeight.com/features/dont-worry-about-the-job-market-yet-but-pay-attention/}
	}
}

\frame{
	\frametitle{Probability in the news}
	There are some interesting fact/analysis raised in this article, e.g.,
	\begin{itemize}
		\item  The unemployment rate fell to 4.3 percent in May, its lowest level since 2001. 
		\item That might sound like good news. It isn’t. Why? \pause
		\item The government only considers people unemployed if they are actively looking for work. \pause
		\item The labor force (everyone who is either working or actively looking for work) shrank by more than 400,000 people.  \pause
		\item Possible comment: naively reading off the unemployment rate drop can be misleading. This article provides a more in-depth view of the other driving force of unemployment rate: size of the labor force. 
	\end{itemize}
}
\frame{
	\frametitle{Probability in the news}
	\begin{itemize}
		\item News from different regions \& fields are welcome!\pause
		\item You can agree or disagree with the news you post. \pause
		\item Comment on the background of the issue so others can understand better. \pause
		\item Engage in discussions. \pause
		\item Be friendly in discussions. \pause
		\item Do not post plain weather forecast, unless the use of probability is very very interesting.\pause
		\item Do not post fictional news (e.g., the onion), unless the use of probability is very very interesting.
	\end{itemize}
}
%================================================================%
\section{History}
\stepcounter{subsection}
\frame{
	\frametitle{Course expectations}
	\begin{itemize}
		\item \blue{What we will learn in this class}
			\begin{itemize}
				\item Probability theories, some statistics, some mathematical tools. \pause
				\item Using the tools of probability and statistics to solve problems. \pause
				\item Using probabilistic reasoning in real life.\pause
			\end{itemize}
		
	\end{itemize}
}
\frame{
	\frametitle{More specifically}
	\begin{itemize}
		\item Gambling! \pause
		\item Not really, but the theory of probability is always associated with it. \pause
		\item Some common tools
				\begin{itemize}
					\item Coin: Heads (H) or Tails (T).
					\item Die: $\{1, 2, 3, 4, 5, 6\}$
					\item Full deck of cards: $52$ cards
				\end{itemize}
		\pause
		\item A typical question you will see in this course:	
	\end{itemize}
				\vspace{-.5cm}
				\begin{quote}
				Start with a shuffled deck of cards and distribute all 52 cards to 4 players, 13 cards to each. What is the probability that each player gets an Ace? 

				Next, assume that you are a player and you get a single Ace. What is the probability now that each player gets an Ace?
			\end{quote}	
}
\frame{
	\frametitle{History of probabilities}
	\begin{itemize}
		\item Modern probability is considered to be born in 1654 when a nobleman wrote a letter to the mathematician and philosopher Blaise Pascal:
	\end{itemize}\pause
	\vspace{-.2cm}
				\begin{quote}
				I used to bet even money that I would get at least one 6 in four rolls of a fair die. The probability of this is 4 times the probability of getting a 6 in a single die, i.e., $4/6 = 2/3$; clearly I had an advantage and indeed I was making money. 

				Now I bet even money that within 24 rolls of two dice I get at least one double 6. This has the same advantage ($24/6^2 = 2/3$), but now I am losing money. Why?
			\end{quote}			
}
\frame{
	\frametitle{History of probabilities}
	\begin{itemize}
		\item This problem starts the correspondence between Pascal and Pierre de Fermat.
		\item And the series of correspondence is credited for the founding of probability theory. \pause
		\item \blue{Now, do you see what's going wrong in the question?} \pause
		\item You will solve this problem in HW1 \Winkey
	\end{itemize}
}
\frame{
	\frametitle{Another example of probabilities}
	\begin{itemize}
		\item In a family with $4$ children, what is the probability of a $2:2$ boy-girl split? \pause
		\item Is it close to $1$, since it is the most ``balanced'' possibility? \pause
		\item Is it $1/5$, since there are 5 possible combinations? \pause
		\item Again, you will solve this problem in HW1 \Winkey
	\end{itemize}
}


\frame{
	\frametitle{A more challenging example}
	\begin{itemize}
		\item Suppose you're on a game show, and you're given the choice of three doors.  \pause
		\item Behind one door is a car; behind the others, goats.  \pause
		\item The host knows where the car is. \pause
		\item You pick a door and the host opens another door and you see there is a goat.  \pause
		\item He then says to you, ``Do you want to change your selection?''
	\end{itemize}
	\pause
	\centering
	\includegraphics[width=.8\textwidth]{fun/monty.png}
}
\frame{
	\frametitle{Another mind twisting example}
	\begin{itemize}
		\item Suppose I have two envelopes with money in them. \pause
		\item One contains twice the money than the other.\pause
		\item I give you one; you open it and see 100 dollars. \pause
		\item Now I say
	\end{itemize}
	\vspace{-.5cm}
	\begin{quote}
	I can give you a chance to swap the envelope. You have $50\%$ chance of gaining another 100 dollars, and $50\%$ chance of losing 50 dollars. So it is to your advantage if you swap.
	\end{quote}\pause
	\vspace{-.5cm}
	\begin{itemize}
		\item Can you spot anything that does not make sense? \pause
		\item No matter what you see in your envelop, you should always swap, so you don't need to open it at all.
		\item What's wrong?
	\end{itemize}
}

\frame{
	\frametitle{Some other things you will see in this course}
	\begin{itemize}
		\item What is the probability that among the $n$ people, no $3$ of them have their birthday on the same day? \pause
		\item What is the probability of seeing at least 5 earthquakes in the next 10 weeks given the average rate at which earthquakes happen?\pause
		\item A man carries 2 matchboxes. Each time he needs a match, he is equally likely to take it from either of them. When he first discovers that one of his matchboxes is empty, what is the probability that there are $k$ matches in the other matchbox?\pause
		\item What is the power of a single voter in a state with $n$ total voters, in terms of electoral college votes?\pause
	\end{itemize}
}

\frame{
	\frametitle{Now let us lay out the actual plan}
	\begin{itemize}
		\item \blue{Today: Combinatorial analysis} \pause
					\begin{itemize}
						\item The fundamentals of probabilistic thinking: counting, permutations, combinations, etc.
					\end{itemize}\pause
		\item \blue{Week 1: Axioms of probability}\pause
			\begin{itemize}
				\item The language we need to communicate probabilities.
			\end{itemize}\pause
		\item \blue{Week 2 \& 3: Conditional probabilities and independence}\pause
				\begin{itemize}
					\item The concepts of probability under partial information. (Remember swapping gate and envelope?)
				\end{itemize} \pause
		\item \blue{Week 3 \& 4: Random variables}		\pause	
					\begin{itemize}
						\item The ability to characterize more complicated problems
					\end{itemize}			
	\end{itemize}
}

\frame{
	\frametitle{Before we get started}
	\begin{itemize}
		\item Probability at the introductory level is not difficult. \pause
		\item But human are typically ``probability blind''.\pause
		\item So, we are going to learn math tools to help us solve problems.\pause
		\item But more importantly, we will train our minds to reason with probabilities, and not to be fooled by randomness.
	\end{itemize}
}

\end{document}