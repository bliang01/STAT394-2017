\documentclass[usenames,dvipsnames,12pt,compress, final]{beamer}\usepackage[]{graphicx}\usepackage[]{color}
\usepackage{alltt} 
\let\Tiny=\tiny
\usepackage{eqnarray,amsmath}
% \usepackage{wasysym}
\usepackage{mathtools}
\usepackage{multirow}
\usepackage{booktabs}
\usepackage{tabularx}
\usepackage{array}
\usepackage{multirow}
% \usepackage{bigstrut}
\usepackage{graphicx}
\usepackage[round]{natbib}
\usepackage{bm}
\usepackage{tikzsymbols}
\setbeamersize{text margin left=15pt,text margin right=10pt}
\usepackage{tikzsymbols}
\usepackage{textcomp}
\usepackage{parskip}
\setbeamertemplate{navigation symbols}{}    
\setbeamertemplate{footline}[frame number]{}
\usepackage{soul}
\usetheme{Singapore}
\usecolortheme{lily}
%Misc Commands
\newcommand{\mbf}{\mathbf}
\newcommand{\lexp}{$\overset{\mbox{\tiny 0}}{e}$}
\newenvironment{wideitemize}{\itemize\addtolength{\itemsep}{5pt}}{\enditemize}


\newcommand{\bx}{{\bm x}}
\newcommand{\bX}{{\bm X}}
\newcommand{\by}{{\bm y}}
\newcommand{\bY}{{\bm Y}}
\newcommand{\bW}{{\bm W}}
\newcommand{\bG}{{\bm G}}
\newcommand{\bR}{{\bm R}}
\newcommand{\bZ}{{\bm Z}}
\newcommand{\bV}{{\bm V}}
\newcommand{\bL}{{\bm L}}
\newcommand{\bz}{{\bm z}}
\newcommand{\be}{{\bm e}}
\newcommand{\bgamma}{{\bm \gamma}}
\newcommand{\bbeta}{{\bm \beta}}
\newcommand{\balpha}{{\bm \alpha}}
\newcommand{\bSigma}{{\bm \Sigma}}
\newcommand{\bmu}{{\bm \mu}}
\newcommand{\btheta}{{\bm \theta}}
\newcommand{\bepsilon}{{\bm \epsilon}}
\newcommand{\bone}{{\bm 1}}
\newcommand{\bzero}{{\bm 0}}
\newcommand{\bC}{{\bm C}}
\newcommand{\bI}{{\bm I}}
\newcommand{\bA}{{\bm A}}
\newcommand{\bB}{{\bm B}}
\newcommand{\bQ}{{\bm Q}}
\newcommand{\bS}{{\bm S}}
\newcommand{\bD}{{\bm D}}
\newcommand{\cQ}{\mathcal{Q}}
\newcommand{\cR}{\mathcal{R}}
\newcommand{\cU}{\mathcal{U}}
\newcommand{\cI}{\mathcal{I}}
\newcommand{\cL}{\mathcal{L}}
\newcommand{\RR}{\mathbb{R}}
\newcommand{\orange}{\textcolor{Orange}}
\newcommand{\green}{\textcolor{green}}
\newcommand{\blue}{\textcolor{blue}}
\newcommand{\red}{\textcolor{red}}
\newcommand{\purple}{\textcolor{purple}}
\newcommand{\gray}{\textcolor{gray}}
\newcommand{\ok}{\nonumber}

% Adjust vertical spacing in lists
\makeatletter
\def\@listi{\leftmargin\leftmargini
            \topsep 		8\p@ \@plus2\p@ \@minus2.5\p@
            \parsep 		0\p@
            \itemsep	5\p@ \@plus2\p@ \@minus3\p@}
\let\@listI\@listi
\def\@listii{\leftmargin\leftmarginii
              \topsep    6\p@ \@plus1\p@ \@minus2\p@
              \parsep    0\p@ \@plus\p@
              \itemsep  3\p@ \@plus2\p@ \@minus3\p@}
\def\@listiii{\leftmargin\leftmarginiii
              \topsep    3\p@ \@plus1\p@ \@minus2\p@
              \parsep    0\p@ \@plus\p@
              \itemsep  2\p@ \@plus2\p@ \@minus3\p@}
\makeatother
% Dealing with fraile envrionment of beamer with codes
\newenvironment{xframe}[2][]
  {\begin{frame}[fragile,environment=xframe,#1]
  \frametitle{#2}}
  {\end{frame}}



\title{Stat 394 Probability I}
\subtitle{Lecture 2}

\author[]{Richard Li}
\date{\today}
\IfFileExists{upquote.sty}{\usepackage{upquote}}{}
\begin{document}
% \renewcommand{\itemize}[1][<+(1)->]{\olditemize[#1]}

\maketitle

%================================================================%
\section{Problem session}
\stepcounter{subsection}
\frame{
	\frametitle{Problem 4}
	In the game of odd man out, each player tosses a fair coin. If all the coins come out the same, except for one, the minority coin is declared "odd man out" and is out of the game. Suppose that three people play odd man out. What is the probability that on the first toss someone will be eliminated? What if they play the same game by rolling a die?
	\vspace{3cm}
}
\frame{
	\frametitle{Problem 5}
	Three balls are dropped at random into three boxes. What is the probability that exactly one box is empty?
	\vspace{5cm}
}
\frame{
	\frametitle{Problem 6}
	Suppose you draw one card at the time from a deck of 52 cards. What is the probability that the second card drawn is an ace?
	\vspace{5cm}
}
\frame{
	\frametitle{Problem 7}
	The probability that there are n insured losses throughout a year obey the rule $p_{n+1} = p_n/5$. What is the probability that there are two or more insured losses?
	\vspace{5cm}
}
\frame{
	\frametitle{Problem 8}
	Without calculation, use logic and definition to explain
		\[
			{n\choose k} = {n-1 \choose k-1} + {n-1 \choose k}
		\]
	\vspace{5cm}	
}
\frame{
	\frametitle{Problem 9}
	Without calculation, use logic and definition to explain
		\[
			k{n\choose k} = n{n-1 \choose k-1}
		\]
	\vspace{5cm}	
}
\frame{
	\frametitle{Problem 10}
	 A woman has $n$ keys, of which one will open her door. If she tries the keys at random, discarding those that do not work, what is the probability that she will open the door on her $k$th try? What if she does not discard any of the tried keys (i.e., in this case, she might try the same key multiple times in a row)?  	
	 \vspace{4cm}
}


%================================================================%
\section{More probability examples}
\stepcounter{subsection}

\frame{
	\frametitle{Inclusion-Exclusion Principle}
}


\frame{
	\frametitle{Example}
	Pick an integer in $[1,1000]$ at random. Compute the probability that it is divisible neither by 12 nor by 15.
	\vspace{5cm}
}
\frame{
	\frametitle{Example}
	Sit 3 men and 4 women at random in a row. What is the probability that either all the men or all the women end up sitting together?
	\vspace{5cm}
}

\frame{
	\frametitle{Example}
	A group of 3 Democrats, 4 Republicans, and 5 Independents is seated at random around a table. Compute the probability that at least one of the three groups ends up sitting together.
	\vspace{5cm}
}
\frame{
	\frametitle{Example}
	A large company with n employees has a scheme according to which each employee buys a Christmas gift and the gifts are then distributed at random to the employees. What is the probability that someone gets his or her own gift?
	\vspace{5cm}
}
\frame{
	\frametitle{Example}
	
}
%================================================================%
\section{Series}
\stepcounter{subsection}
\frame{
	\frametitle{Just a review}
	A good time to recall some common mathematical series:
	\begin{eqnarray}\ok
		\exp(x)&=& \sum_{n = 0}^\infty \frac{x^n}{n!}
	\\\ok
		log(1 - x)&=& -\sum_{n = 1}^\infty \frac{x^n}{n}
	\\\ok
		log(1 + x) &=& \sum_{n = 1}^\infty (-1)^{n+1} \frac{x^n}{n}
	\\\ok
		\frac{1}{1 - x} &=& \sum_{n = 0}^\infty x^n
	\end{eqnarray}
}
\frame{
	\frametitle{Binomial series, or binomial theorem}
	\[
		(x + y)^n = \sum_{k = 0}^n {n \choose k} x^k y^{n - k}
	\]
	\vspace{5cm}


}
\frame{
	\frametitle{Binomial series, or binomial theorem}


}

\frame{
	\frametitle{Example}
	How many subsets are there of a set consisting of $n$ elements?
	\vspace{5cm}
}

\frame{
	\frametitle{Example: Birthday Problem}
	 Assume that there are k people in the room. What is the probability that there are two who share a birthday? We will ignore leap years, assume all birthdays are equally likely.
	\vspace{5cm}
}

\frame{
	\frametitle{More on the birthday problem}
	\begin{itemize}
		\item Each day, the Massachusetts lottery chooses a four digit number at random, with leading 0's allowed. 
		\item On February 6, 1978, the Boston Evening Globe reported that
	\end{itemize}
	\begin{quote}
			``During [the lottery’s] 22 months’ existence [...], no winning number has ever been repeated. [...] doesn’t expect to see duplicate winners until about half of the 10, 000 possibilities have been exhausted.''
	\end{quote}
	\begin{itemize}
		\item What if $k$ people enter the room one by one, which person has the highest probability of being the first to share the same birthday with the people in the room? 
	\end{itemize}
}

\frame{
	\frametitle{Coupon collector problem}
	Within the context of the birthday problem, assume that $k \geq n$ and compute $P(\mbox{all n birthdays are represented})$.
	\vspace{5cm}
}

\frame{
	\frametitle{Coupon collector problem}
	
}
\end{document}